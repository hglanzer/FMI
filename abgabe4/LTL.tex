\Aufgabe[LTL - Monotonicity and Negation Normal Form \hfill\textbf{(2 Point)}]

\begin{enumerate}

\item Let $K_1 = (S, R, L_1)$ and  $K_2 = (S, R, L_2)$ be two Kripke structures with the same set of states $S$ and the same transition relation $R$ such that $L_1(s) \subseteq L_2(s)$ for all states $s \in S$.
    Prove that $K_1,s \models \phi$ implies $K_2,s \models \phi$ for all LTL formulae $\phi$ that do not contain negation.
    \emph{(Hint: prove this statement by structural induction)}.

\item

Exercise 1 defined the \emph{release operator} \textbf{R}.
Prove that the release operator enjoys the following equivalence using the semantics of LTL \emph{(Hint: use the semantics of LTL formulae)}:
\begin{displaymath}
    \phi \mathbf{R} \psi \equiv \neg(\neg \psi \mathbf{U} \neg \phi)
\end{displaymath}

\item

An LTL formula in \emph{negation normal form}, if
\begin{itemize}
\item all negations appear only in front of the atomic propositions,
\item only the logical operators \emph{true}, \emph{false}, $\vee$, and $\wedge$ are used, and
\item only the temporal operators \textbf{X}, \textbf{U}, and \textbf{R} are used.
\end{itemize}

Show that every LTL formula $\phi$ can be transformed into an equivalent formula $\psi$ that is in negation normal form.
\emph{(Hint: prove this statement by structural induction)}.

\end{enumerate}
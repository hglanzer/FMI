\Aufgabe[LTL - Monotonicity and Negation Normal Form \hfill\textbf{(2 Point)}]

\begin{enumerate}

\item Let $K_1 = (S, R, L_1)$ and  $K_2 = (S, R, L_2)$ be two Kripke structures with the same set of states $S$ and the same transition relation $R$ such that $L_1(s) \subseteq L_2(s)$ for all states $s \in S$.
    Prove that $K_1,s \models \phi$ implies $K_2,s \models \phi$ for all LTL formulae $\phi$ that do not contain negation.
    \emph{(Hint: prove this statement by structural induction)}.

\item

Exercise 1 defined the \emph{release operator} \textbf{R}.
Prove that the release operator enjoys the following equivalence using the semantics of LTL \emph{(Hint: use the semantics of LTL formulae)}:
\begin{displaymath}
\phi R \psi \equiv \neg (\neg \phi U \neg \psi)
    %\phi \mathbf{R} \psi \equiv \neg(\neg \psi \mathbf{U} \neg \phi)
\end{displaymath}

We start at the definition of the until operator:\\
\\
$\phi U \psi = \exists j \geq 0 (\pi^{j} \vDash \psi \wedge \forall i<j, \pi^{i} \vDash \phi)$\\
\\
We insert the negative operators:\\
\\
$\neg (\neg \phi U \neg \psi) = \neg \exists j \geq 0 (\pi^{j} \nvDash \psi \wedge \forall i<j, \pi^{i} \nvDash \phi)$\\
\\
$\forall j \geq 0 \neg(\pi^{j} \nvDash \psi \wedge \forall i<j, \pi^{i} \nvDash \phi)$\\
$\forall j \geq 0 (\neg(\pi^{j} \nvDash \psi) \vee \neg\forall i<j, \pi^{i} \nvDash \phi)$\\
$\forall j \geq 0 (\pi^{j} \vDash \psi \vee \exists i<j, \pi^{i} \vDash \phi)$\\
\\
At this step we need to split the left from the right side of the or operator. That's why we need to use the variable i with the same use as j 
and define that it must be bigger then zero. The new introduced variable k has the same purpose as i before, which must be smaller then i.\\
\\
$\forall j \geq 0 \pi^{j} \vDash \psi 	or	 \exists i\geq 0, \pi^{i} \vDash \phi \wedge \forall k\leq i,\pi^{k} \nvDash \psi$\\
\\
At this point we have reached the definition of the replace operator and have proven, that $\phi R \psi \equiv \neg (\neg \phi U \neg \psi)$ is valid.


\item

An LTL formula in \emph{negation normal form}, if
\begin{itemize}
\item all negations appear only in front of the atomic propositions,
\item only the logical operators \emph{true}, \emph{false}, $\vee$, and $\wedge$ are used, and
\item only the temporal operators \textbf{X}, \textbf{U}, and \textbf{R} are used.
\end{itemize}

Show that every LTL formula $\phi$ can be transformed into an equivalent formula $\psi$ that is in negation normal form.
\emph{(Hint: prove this statement by structural induction)}.

\end{enumerate}
\documentclass[a4paper]{scrartcl}
\usepackage{amsmath,amssymb,fvsw}
\newtheorem{ex}{Exercise}
\newenvironment{exercise}[1]%
   {\def\tmp{}%
    \def\points{#1}%
    \ifx\points\tmp
      \begin{ex}
    \else
      \def\tmp{1}%
    \begin{ex}[#1 point\ifx\points\tmp\else s\fi]
    \fi
    \normalfont
   }%
   {\end{ex}%
   }

\usepackage{enumerate}
\newenvironment{subexercises}%
   {\begin{enumerate}[(a)]}%
   {\end{enumerate}}

\renewcommand\paragraph[1]{\medskip\par\noindent\textit{#1}}

\begin{document}
\title{Formal Verification of Software~-- Exercises}
\date{May 2012}
\author{First Author\and Second Author\and Third Author}
\maketitle

\begin{exercise}{1}\label{ex:syntax}
  Show that the \TPL\ program given in exercise~\ref{ex:ifabort}
  is syntactically correct.
\end{exercise}

\begin{exercise}{1}
  Let $\sigma$ be a state satisfying $\sigma(x)=\sigma(y)=1$, and let
  $p$ be the program given in exercise~\ref{ex:ifabort}.
  Compute $\M p\sigma$, using
  \begin{subexercises}
    \item the structural operational semantics
    \item the natural semantics
  \end{subexercises}
  of \TPL.
\end{exercise}

\begin{exercise}{1}\label{ex:ifabort}
  Let $p$ be the following program:
  \begin{center}
  \begin{ALG}
    \ASS x{x+y};\\
    \IF\ $x<0$ \THEN\\
    \>\ABORT\\
    \ELSE\\
    \>\WHILE\ $x\neq y$ \DO\\
    \>\>\ASS x{x+1};\\
    \>\>\ASS y{y+2}\\
    \>\OD\\
    \FI
  \end{ALG}
  \end{center}
  Show that $\CA{x=2y\land y>2}p{x=y}$ is totally correct by computing the
  weakest precondition of the program.
\end{exercise}

\begin{exercise}{1}
  Let $p$ be the program given in exercise~\ref{ex:ifabort}.
  Use the Hoare calculus to show that
  \[\CA{x=2y\land y>2}p{x=y}\]
  is totally correct.
\end{exercise}

\begin{exercise}{1}
  Extend our toy language by statements of the form ``$\kw{assert}\
  e$''. When the condition~$e$ evaluates to true, the program
  continues, otherwise the program aborts.

  Specify the syntax and semantics of the extended language.
  Determine the weakest precondition, the weakest liberal
  precondition, the strongest postcondition, and Hoare rules
  (partial and total correctness) for \kw{assert}-statements.
  Show that they are correct.

  Treat the \kw{assert}-statement as a first-class citizen, i.e., do
  not refer to other program statements in the final result.  However,
  you may use other statements as intermediate steps when deriving the
  rules.
\end{exercise}

\begin{exercise}{1}
  Verify that the following program doubles the value of~$x$.
  For which inputs does it terminate? Choose appropriate pre-
  and postconditions and show that the assertion is totally correct.
  Use $y=2x_0+x$ as a starting point for the invariant,
  where $x_0$ denotes the initial value of~$x$.
\begin{center}
  \begin{ALG}
    \ASS y{3x};\\
    \WHILE\ $2x\neq y$\ \DO\\
    \>\ASS x{x+1};\\
    \>\ASS y{y+1};\\
    \OD
  \end{ALG}
\end{center}    
\end{exercise}

\begin{exercise}{1}
    Show that the following correctness assertion is totally correct.
    Describe the function computed by the program if we consider $a$
    as its input and $c$ as its output.
\begin{center}
  \begin{ALG}
    \ASSERTN1{a\geq0}\\
    \ASS b1;\\
    \ASS c0;\\
    \ASSERTN\INV{b=(c+1)^3\land 0\leq c^3\leq a}\\
    \WHILE\ $b\leq a$\ \DO\\
    \>\ASS d{3*c+6};\\
    \>\ASS c{c+1};\\
    \>\ASS b{b+c*d+1}\\
    \OD\\
    \ASSERTN2{c^3\leq a<(c+1)^3}
  \end{ALG}
\end{center}    
\end{exercise}

\begin{exercise}{1}
  Prove that the rule
\[
\begin{array}{c}
\WH{\CA{\INV}{\WHILE\ e\ \DO\ p\ \OD}{\INV\land\lnot e}}%
   {\CA{\INV\land e}{p}{\INV}}
\end{array}
\]
is correct regarding partial correctness, i.e., show that
$\CA{\INV}{\WHILE\ e\ \DO\ p\ \OD}{\INV\land\lnot e}$ is partially correct
whenever $\CA{\INV\land e}{p}{\INV}$ is partially correct.
\end{exercise}

\begin{exercise}{2}
  Determine the weakest liberal precondition of \WHILE-loops, i.e.,
  find a formula equivalent to $\WLP(\WHILE\ e\ \DO\ p\ \OD, G)$
  similar to the weakest precondition in the course.

  Use your formula to compute the weakest liberal precondition of
  the program 
  \begin{center}
     \ASS z0;
     \WHILE\ $y\neq0$ \DO
       \ASS z{z+x};
       \ASS y{y-1}
     \OD
  \end{center}
  with respect to the postcondition $z=x*y_0$. Compare the result
  to the weakest precondition computed in the course and explain
  the differences.
\end{exercise}


\end{document}

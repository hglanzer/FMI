\documentclass[a4paper]{scrartcl}
\usepackage{amsmath,amssymb,fvsw}
\newtheorem{ex}{Exercise}
\newenvironment{exercise}[1]%
   {\def\tmp{}%
    \def\points{#1}%
    \ifx\points\tmp
      \begin{ex}
    \else
      \def\tmp{1}%
    \begin{ex}[#1 point\ifx\points\tmp\else s\fi]
    \fi
    \normalfont
   }%
   {\end{ex}%
   }

\usepackage{enumerate}
\newenvironment{subexercises}%
   {\begin{enumerate}[(a)]}%
   {\end{enumerate}}

\renewcommand\paragraph[1]{\medskip\par\noindent\textit{#1}}

\begin{document}
\title{Formal Verification of Software~-- Exercises}
\date{May 2012}
\author{Harald Glanzer\and Bernd-Peter Ivanschitz\and Lukas Petermann}
\maketitle

\begin{exercise}{1}\label{ex:syntax}
  Show that the given\TPL\ program 
  is syntactically correct:
\newline

  \begin{ALG}
    1    \ASS x{x+y};\\
    2    \IF\ $x<0$ \THEN\\
    3    \>\ABORT\\
    4    \ELSE\\
    5    \>\WHILE\ $x\neq y$ \DO\\
    6    \>\>\ASS x{x+1};\\
    7    \>\>\ASS y{y+2}\\
    8    \>\OD\\
    9    \FI
  \end{ALG}


To prove the progam for correct syntax, every line of the given code must be examined to show that every statement is part of our programming language. The TPL
syntax was introduced in the first lecture, see the corresping pdf, pages 12 and 13.

\begin{itemize}
 \item Line 1: \begin{ALG} \ASS x{x+y}; \end{ALG} \newline
Assignment: $\nu::= \epsilon, \nu::= variable | integer | binary$ $expression | unary$ $expression$

Left side of an assignment must be a variable, which must NOT be a keyword, which is not the case, $x$ is not reserved, as well as $y$. Left 
side must be an expression $\epsilon$, which can itself be another variable, an integer or one or two expressions, connected with an uniary or
binary operator -> syntactically correct

 \item Line 2, 4, 9: \begin{ALG} \IF\ $x<0$ \THEN \end{ALG} ... \begin{ALG} else \end{ALG} ... \begin{ALG} fi \end{ALG} \newline
If-Then-Else - Statement: must start with the keyword $IF$, followed by an expression $\epsilon$. In this case, the expression consists
of 2 variables, connected with a binary operator, as defined recursevly in the lecture notes. After this expression another keyword, $THEN$, must occur,
which is the case. A program - statement must follow afterwards(will be examined in the next steps), until another keyword, $ELSE$, occurs. Afterwards, another
program statement follows, until the construct is closed by the final keyword $FI$.

 \item Line 3: \begin{ALG} abort \end{ALG} \newline
valid keyword, which terminates the actual program and signals an error condition
 \item Line 5,8: \begin{ALG} \>\WHILE\ $x\neq y$ \DO \end{ALG} ...   \begin{ALG} od \end{ALG} \newline
keyword $WHILE$, followed by a valid binary expression, followd by keyword $DO$, followed by a program statement, terminated by keyword $OD$
 \item Line 6: \begin{ALG}  \>\>\ASS x{x+1}; \end{ALG} \newline
valid assignment, left side is a variable(no keyword allowed), right side consists of a binary expression, in this case a variable and an integer, which
is a valid expression as defined in a recursive manner on page 12 in the lecture notes 

 \item Line 7: \begin{ALG}  \>\>\ASS y{y+2};\end{ALG}\newline
same as Line 6

\end{itemize}


\end{exercise}

\begin{exercise}{1}
  Let $\sigma$ be a state satisfying $\sigma(x)=\sigma(y)=1$, and let
  $p$ be the program given in exercise~\ref{ex:ifabort}.
  Compute $\M p\sigma$, using
  \begin{subexercises}
    \item the structural operational semantics
    \item the natural semantics
  \end{subexercises}
  of \TPL.
\end{exercise}

\begin{exercise}{1}\label{ex:ifabort}
  Let $p$ be the following program:
  \begin{center}
  \begin{ALG}
    \ASS x{x+y};\\
    \IF\ $x<0$ \THEN\\
    \>\ABORT\\
    \ELSE\\
    \>\WHILE\ $x\neq y$ \DO\\
    \>\>\ASS x{x+1};\\
    \>\>\ASS y{y+2}\\
    \>\OD\\
    \FI
  \end{ALG}
  \end{center}
  Show that $\CA{x=2y\land y>2}p{x=y}$ is totally correct by computing the
  weakest precondition of the program.
\end{exercise}

\begin{exercise}{1}
  Let $p$ be the program given in exercise~\ref{ex:ifabort}.
  Use the Hoare calculus to show that
  \[\CA{x=2y\land y>2}p{x=y}\]
  is totally correct.
\end{exercise}

\begin{exercise}{1}
  Extend our toy language by statements of the form ``$\kw{assert}\
  e$''. When the condition~$e$ evaluates to true, the program
  continues, otherwise the program aborts.

  Specify the syntax and semantics of the extended language.
  Determine the weakest precondition, the weakest liberal
  precondition, the strongest postcondition, and Hoare rules
  (partial and total correctness) for \kw{assert}-statements.
  Show that they are correct.

  Treat the \kw{assert}-statement as a first-class citizen, i.e., do
  not refer to other program statements in the final result.  However,
  you may use other statements as intermediate steps when deriving the
  rules.
\end{exercise}

\begin{exercise}{1}
  Verify that the following program doubles the value of~$x$.
  For which inputs does it terminate? Choose appropriate pre-
  and postconditions and show that the assertion is totally correct.
  Use $y=2x_0+x$ as a starting point for the invariant,
  where $x_0$ denotes the initial value of~$x$.
\begin{center}
  \begin{ALG}
    \ASS y{3x};\\
    \WHILE\ $2x\neq y$\ \DO\\
    \>\ASS x{x+1};\\
    \>\ASS y{y+1};\\
    \OD
  \end{ALG}
\end{center}    
\end{exercise}

\begin{exercise}{1}
    Show that the following correctness assertion is totally correct.
    Describe the function computed by the program if we consider $a$
    as its input and $c$ as its output.
\begin{center}
  \begin{ALG}
    \ASSERTN1{a\geq0}\\
    \ASS b1;\\
    \ASS c0;\\
    \ASSERTN\INV{b=(c+1)^3\land 0\leq c^3\leq a}\\
    \WHILE\ $b\leq a$\ \DO\\
    \>\ASS d{3*c+6};\\
    \>\ASS c{c+1};\\
    \>\ASS b{b+c*d+1}\\
    \OD\\
    \ASSERTN2{c^3\leq a<(c+1)^3}
  \end{ALG}
\end{center}    
\end{exercise}

\begin{exercise}{1}
  Prove that the rule
\[
\begin{array}{c}
\WH{\CA{\INV}{\WHILE\ e\ \DO\ p\ \OD}{\INV\land\lnot e}}%
   {\CA{\INV\land e}{p}{\INV}}
\end{array}
\]
is correct regarding partial correctness, i.e., show that
$\CA{\INV}{\WHILE\ e\ \DO\ p\ \OD}{\INV\land\lnot e}$ is partially correct
whenever $\CA{\INV\land e}{p}{\INV}$ is partially correct.
\end{exercise}

\begin{exercise}{2}
  Determine the weakest liberal precondition of \WHILE-loops, i.e.,
  find a formula equivalent to $\WLP(\WHILE\ e\ \DO\ p\ \OD, G)$
  similar to the weakest precondition in the course.

  Use your formula to compute the weakest liberal precondition of
  the program 
  \begin{center}
     \ASS z0;
     \WHILE\ $y\neq0$ \DO
       \ASS z{z+x};
       \ASS y{y-1}
     \OD
  \end{center}
  with respect to the postcondition $z=x*y_0$. Compare the result
  to the weakest precondition computed in the course and explain
  the differences.
\end{exercise}


\end{document}
